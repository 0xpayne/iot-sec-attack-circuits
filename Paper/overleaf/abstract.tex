\begin{abstract}
The proliferation of IoT Devices is wide-spread and continuing to increase in a superlinear manner. As a result, the complexity and commonality of cybersecurity challenges in new and dynamic environments are rapidly increasing. The following question arises: how can an IoT network administrator determine the network's vulnerabilities, as well as the extent of the network's holistic security risk, especially given the increasing complexity of IoT networks? In this paper, we propose and implement the notion of an \textit{attack circuit}: a system and set of techniques for the holistic assessment of cybersecurity risks for a network of IoT devices. Our system provides a practical method of finding answers to the aforementioned questions and gives insight into the possible attack paths an adversary may utilize. We also propose the use of two types of risk measures, \textit{compositional scores} and \textit{dynamic activity metrics}, which are used to evaluate IoT devices and the overall IoT network in the context of an attack circuit, and demonstrate the effectiveness of attack circuits as practical tools for computing these scores as well as finding optimal attack paths (by metrics of exploitability, impact, and risk to confidentiality, integrity, and availability) in heterogeneous, extensible IoT networks.
\end{abstract}
