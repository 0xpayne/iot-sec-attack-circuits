\section{Conclusion}
\label{sec:conclusion}

In this paper, we address the problem of evaluating the security of a network. This is done by using \textit{attack circuits} and associated compositional scores and dynamic activity metrics. In this manner, an individual IoT device or network may be analyzed for its vulnerability to security attacks. Evaluation in Section \ref{sec:evaluation} demonstrates the increased security risks for a growing IoT network. While our work focuses on using descriptions to extract \textit{input/output} pairs, this approach may be extended to extract multiple pairs per description, as well as using other available information sources. Activity metrics may be further developed by using generative machine learning models to learn abnormal network traffic behaviors. We also noted that as networks grow, their complexity grows exponentially. Thus, the network flow problems described may become inefficient with huge IoT attack circuits, and alternative approaches like graph neural networks (GNNs) \cite{Scarselli09thegraph} may be required for their analysis. However, for application in smaller IoT networks (e.g. smart homes), we conclude that this approach suffices.

%we use 3 metrics (for now)

% adding network-particular score relativizing as future work

