\documentclass[10pt, conference]{IEEEtran}

%some packages we might use
\usepackage{subfigure}
\usepackage{graphicx}
\usepackage{graphicx, subfigure, epsfig, multirow}
% \usepackage[urlbordercolor={1 1 1}, linkbordercolor={1 1 1}]{hyperref}
\usepackage[urlbordercolor={1 1 1} ]{hyperref}
\usepackage[hyphenbreaks]{breakurl}
\usepackage{amssymb,amsmath}
% \usepackage[ruled,vlined]{algorithm2e}
\usepackage{algorithm}% http://ctan.org/pkg/algorithms
\usepackage{algpseudocode}% http://ctan.org/pkg/algorithmicx
\usepackage{color}

%this makes our life easy
\newcommand{\tabincell}[2]{\begin{tabular}{@{}#1@{}}#2\end{tabular}}
\newcommand{\code}[1]{{\fontfamily{cmtt}\fontseries{m}\fontshape{n}\selectfont\footnotesize{#1}}}
\newcommand{\captionsmall}[1]{{\selectfont\small{#1}}}

\newcommand{\ks}[1]{\textcolor{red}{[KS: #1]}}
\newcommand{\TODO}[1]{\textcolor{red}{#1}}
\newcommand{\todo}[1]{\textcolor{red}{#1}}

%change font size in algorithm
\makeatletter
\renewcommand{\ALG@beginalgorithmic}{\footnotesize}
\makeatother

\begin{document}
%
% paper title
\title{Attack Circuits for the Security Evaluation of Smart Homes and other IoT Networks}


\author{ }

% make the title area
\maketitle

\begin{abstract}
The proliferation of IoT Devices is wide-spread and continuing to increase in a superlinear manner. As a result, the complexity and commonality of cybersecurity challenges in new and dynamic environments are rapidly increasing. The following question arises: how can an IoT network administrator determine the network's vulnerabilities, as well as the extent of the network's holistic security risk, especially given the increasing complexity of IoT networks? In this paper, we propose and implement the notion of an \textit{attack circuit}: a system and set of techniques for the holistic assessment of cybersecurity risks for a network of IoT devices. Our system provides a practical method of finding answers to the aforementioned questions and gives insight into the possible attack paths an adversary may utilize. We also propose the use of two types of risk measures, \textit{compositional scores} and \textit{dynamic activity metrics}, which are used to evaluate IoT devices and the overall IoT network in the context of an attack circuit, and demonstrate the effectiveness of attack circuits as practical tools for computing these scores as well as finding optimal attack paths (by metrics of exploitability, impact, and risk to confidentiality, integrity, and availability) in heterogeneous, extensible IoT networks.
\end{abstract}

 

\section{Overview}

Overview
%\input{relatedworks}
%\section{Conclusion}
\label{sec:conclusion}

In this paper, we address the problem of evaluating the security of a network. This is done by using \textit{attack circuits} and associated compositional scores and dynamic activity metrics. In this manner, an individual IoT device or network may be analyzed for its vulnerability to security attacks. Evaluation in Section \ref{sec:evaluation} demonstrates the increased security risks for a growing IoT network. While our work focuses on using descriptions to extract \textit{input/output} pairs, this approach may be extended to extract multiple pairs per description, as well as using other available information sources. Activity metrics may be further developed by using generative machine learning models to learn abnormal network traffic behaviors. We also noted that as networks grow, their complexity grows exponentially. Thus, the network flow problems described may become inefficient with huge IoT attack circuits, and alternative approaches like graph neural networks (GNNs) \cite{Scarselli09thegraph} may be required for their analysis. However, for application in smaller IoT networks (e.g. smart homes), we conclude that this approach suffices.

%we use 3 metrics (for now)

% adding network-particular score relativizing as future work



%the bib
%\footnotesize
\bibliographystyle{plain}
\bibliography{paper_ref}


\end{document}
