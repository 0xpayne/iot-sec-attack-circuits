\section{Introduction}

Statista estimates that the number of connected IoT devices will rise from 19.4B in 2018 to 34.2B in 2025 \cite{statista2019smart}, and that the percentage of U.S. homes that are ``smart" will rise from 33.2\% in 2019 to 53.9\% in 2023 \cite{statista2019internet}. Usage of Internet of Things (IoT) devices in networks such as smart homes, smart cities, and digital healthcare is clearly increasing, and while the adaptable and heterogeneous features of these networks have proven to be crucial in solving a vast array of different issues, they have also given adversaries a veritable sandbox of vulnerabilities to exploit. Security measures that prevent attackers from exploiting these vulnerabilities are more important now than ever before because of the growing quantity and sensitivity of data that people are putting online. The complexity multi-stage privacy, service, and safety attacks is also growing, and it is now critical for home owners, enterprises, or government organizations that host various IoT devices on complex IoT networks (augmented with the Bring Your Own Device (BYOD) trend of devices) to be aware of the cybersecurity risks that may be present. In particular, it is vital to determine high-priority vulnerabilities that need to be addressed in order to keep the overall network, as well as the individual devices, protected.

We propose the notion of an \textit{attack circuit}, a structure that arises very naturally from the often extensible, modular, and heterogeneous nature of IoT networks and helps to model possible attack paths and evaluate the security state of the represented IoT network as well as each individual device therein. Current methods, discussed in Section \ref{sec:related_work}, seek to address these problems as well for general networks as well as specifically for IoT networks--our work builds off of these, bringing ideas from this work together as well as adding our own novelties to address some of the limitations of current work. These novelties solve problems such as vulnerability documentation processing, delivering different types of metrics in the form of \textit{compositional scoring} and network flow analysis, and incorporating network activity data into \textit{dynamic activity metrics} (this is explored further in Section \ref{sec:problem_definition}). In Section \ref{sec:proposed_method}, we discuss the practical and theoretical usefulness of this notion in the context of a smart home, using vulnerability data from the National Vulnerability Database (NVD)\footnote{http://nvd.
nist.gov/download.cfm}, network traffic data from off-the-shelf IoT devices, and optimization and machine learning techniques for constructing and evaluating the resulting attack circuits and attack paths. Finally, in Section \ref{sec:evaluation}, we demonstrate our own implementation of the attack circuit and evaluate the effectiveness of such a notion through quantitative experimentation. This is followed by concluding remarks in Section \ref{sec:conclusion}.

%discuss possible applications of graph learning to the temporal data created by IoT network activity logs, as well as the attack circuit itself, which, as information about the network's devices becomes more available and the network size grows, becomes increasingly hard to evaluate.

% The remainder of the paper is organized as follows. Section \ref{sec:related_work} discusses relevant work in context of our problem. The problem is then formalized in Section \ref{sec:problem_definition} and is followed by an overview of the proposed solution in Section \ref{sec:proposed_method}. Section \ref{sec:implementation} describes the specific details used by our implementation of the proposed solution. Our method is evaluated in Section \ref{sec:evaluation}. This is followed by concluding remarks in Section \ref{sec:conclusion}.